\label{sec:technical}
Below is a a brief description of how the solution works. The
middleware is in {\tt middleware.py}, and uses only the {\tt
  process\_view} method:
\begin{itemize}
\item \emph{Modification of views: } The views in {\tt
    apps.files.views} have been modified, so that for every view
  function, say {\tt view\_func} which is protected by a
  \@login\_required decorator, the decorator is removed, and ``\_reg''
  appended to the function name, e.g., {\tt view\_func\_reg}. The
  normal function is then set up again with
  \begin{Verbatim}
    view_func = login_required( view_func_reg )
  \end{Verbatim}
  Thus, the original functions are protected as before, and the
  new set of XXX\_reg functions are not externally accessible as
  there are no URLs associated with them. The sole exception is
  the {\tt login\_reg} view which is used for logging in users from
  registered IP addresses. Access to this for users from
  non-registered IP addresses is prohibited in the middleware. 
\item \emph{Login templates: } The middleware intercepts the
  login views for users from both registered and unregistered IP
  addresses, and provides the appropriate login template for
  each.
\item \emph{Denying access to {\tt login\_reg} for users from
    unregistered IPs: } The {\tt login\_reg} view is restricted to users
  from registered IP addresses. Requests for {\tt login\_reg}
  view from unregistered IP addresses are redirected to {\tt login}.
\item \emph{Handling of non-admin views: } Non-admin. views from
  users coming from registered IP addresses are shown directly. A
  small complication here is that some views are from {\tt
    apps.files.views} and others from {\tt
    django.contrib.auth.admin.views}, so that the appropriate one
  needs to be found. Also, a small trick is that we directly call
  the view function from {\tt viewfunc.\_\_name\_\_} whereas we
  are starting with an unregistered view. This works because we
  had set:
  \begin{Verbatim}
    view_func = login_required( view_func_reg )
  \end{Verbatim}
  This assignment does not change {\tt viewfunc.\_\_name\_\_}, so
  what we are really calling is the registered function, i.e.,
  {\tt view\_func\_reg} rather than {\tt view\_func}. This is a
  small matter, and we could have easily appended ``\_reg'' if
  needed.
\item \emph{Handling of other views: } All other views are simply
  passed on, so that they are handled by the normal Django
  authentication layer. This includes admin. views from registered
  IP addresses, and all views from unregistered addresses.
\end{itemize}
